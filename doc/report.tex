\documentclass{article}
\usepackage{booktabs}
\usepackage{float}
\usepackage{fancyhdr}
\usepackage{amsmath}
% \usepackage[]{biblatex}
% \usepackage[nottoc]{tocbibind}
\fancyhf{} % Clear header/footer
\renewcommand{\headrulewidth}{0pt} % Remove header rule
\usepackage[english]{babel}
%Includes "References" in the table of contents
% \addbibresource{bib}
% \AtBeginBibliography{\small}
\usepackage{appendix}

\begin{document}

\title{Bibsearch Engine Report}
\author{Your Name}
\date{\today}

\maketitle
\newpage
\tableofcontents

\newpage


\section{Introduction}

% Provide an introduction to the bibsearch engine and its purpose.

% \chapter{Directory Structure}

% \subsection{Root Directory}

% \begin{itemize}
%     \item \texttt{Makefile}: Makefile for building the project.
%     \item \texttt{\_\_init\_\_.py}: Initialization file.
%     \item \texttt{api\_requester.py}: Module for making API requests.
%     \item \texttt{build}: Build directory.
%     \item \texttt{classes.dot}: Graphviz DOT file for class diagrams.
%     \item \texttt{classes\_DEEPCHEM.png}: Image file for class diagrams.
%     \item \texttt{config.ini}: Configuration file in INI format.
%     \item \texttt{config.json}: Configuration file in JSON format.
%     \item \texttt{configurations}: Directory for configuration-related modules.
%     \begin{itemize}
%         \item \texttt{\_\_init\_\_.py}: Initialization file.
%         \item \texttt{read\_config.py}: Module for reading the configuration.
%     \end{itemize}
%     \item \texttt{db}: Directory for database-related modules.
%     \begin{itemize}
%         \item \texttt{\_\_init\_\_.py}: Initialization file.
%         \item \texttt{connection.py}: Module for database connection.
%     \end{itemize}
%     \item \texttt{distributed}: Directory for distributed computing modules.
%     \begin{itemize}
%         \item \texttt{Trie.py}: Module for the Trie data structure.
%         \item \texttt{config\_manager.py}: Module for managing configuration in a distributed environment.
%         \item \texttt{fast\_json\_loader.py}: Module for fast JSON loading.
%         \item \texttt{node.py}: Module for representing a distributed computing node.
%         \item \texttt{node\_data}: Directory for storing node-specific data.
%         \item \texttt{request\_wrapper.py}: Module for wrapping API requests in a distributed environment.
%         \item \texttt{test}: Directory for test-related modules.
%         \item \texttt{utils.py}: Utility module for distributed computing.
%     \end{itemize}
%     \item \texttt{docs}: Directory for documentation.
%     \begin{itemize}
%         \item \texttt{index.md}: Markdown file for the documentation index.
%     \end{itemize}
%     \item \texttt{error\_handling}: Directory for error handling modules.
%     \begin{itemize}
%         \item \texttt{\_\init\_\_.py}: Initialization file.
%         \item \texttt{exceptions.py}: Module for custom exceptions.
%     \end{itemize}
%     \item \texttt{fast\_api.py}: FastAPI module for the web application.
%     \item \texttt{indexer}: Directory for indexing modules.
%     \begin{itemize}
%         \item \texttt{\_\init\_\_.py}: Initialization file.
%         \item \texttt{\_\_pycache\_\_}: Directory for cached Python files.
%         \item \texttt{data\_reader.py}: Module for reading data for indexing.
%         \item \texttt{index\_creator.py}: Module for creating indexes.
%         \item \texttt{index\_metadata.py}: Module for indexing metadata.
%         \item \texttt{index\_valueInfo.py}: Module for indexing value information.
%     \end{itemize}
%     \item \texttt{main.py}: Main entry point for the bibsearch engine.
%     \item \texttt{make.bat}: Batch file for building the project (Windows).
%     \item \texttt{mkdocs.yml}: Configuration file for MkDocs documentation.
%     \item \texttt{preprocessor}: Directory for data preprocessing modules.
%     \begin{itemize}
%         \item \texttt{\_\_init\_\_.py}: Initialization file.
%         \item \texttt{\_\_pycache\_\_}: Directory for cached Python files.
%         \item \texttt{data\_cleaner.py}: Module for cleaning data before indexing.
%     \end{itemize}
%     \item \texttt{pyproject.toml}: Project metadata file.
%     \item \texttt{ranker}: Directory for ranking modules.
%     \begin{itemize}
%         \item \texttt{\_\_init\_\_.py}: Initialization file.
%         \item \texttt{ranking\_algorithms.py}: Module for ranking algorithms.
%         \item \texttt{relevancy.py}: Module for calculating relevancy scores.
%         \item \texttt{search.py}: Module for searching and ranking results.
%     \end{itemize}
% \end{itemize}
\section{Idea}

We have embarked on the ambitious task of developing a bibliographic search engine that 
harnesses the vast amount of data available in the Crossref dataset. With a staggering size of 
% TODO add exact size
157 GB compressed (1 TB uncompressed), this dataset encompasses publication metadata from 
approximately 134 million publications, including full citation data for 60 million of them. Our 
objective was to create a tool that enables users to efficiently explore this extensive 
collection of scholarly resources, and to build a system that can be scaled to handle even larger
datasets in the future.

To handle the scale and complexity of this dataset, we designed our search engine as a 
distributed system, leveraging the power of eight nodes. One node serves as the front-end, 
handling user queries and managing the user interface, while the remaining seven nodes function 
as the back-end, responsible for data storage and retrieval. In order to optimize query 
performance, we implemented a MongoDB database to store the index containing only the 
essential information of each zipped json file such as title, 
abstract, and URL in memory, ensuring rapid access and retrieval. Eventually, 
MongoDB is used to store metadata of each entry. 
% containing metadata of the index 

In addition to establishing a robust infrastructure, we implemented a clear ranking 
system to enhance the relevance and accuracy of search results.
Our ranking algorithm considers was built based on BM25F, %TODO add citation
a widely established ranking algorithm based on the TF-IDF model. This algorithm takes into
account the input query and match as many words from title and abstract. 
Furthermore, an indexing mechanism has been developed
to facilitate quick search operations across the vast dataset, enabling users to obtain relevant 
results within seconds.

To further enhance the user experience and provide valuable suggestions, we integrated a machine 
learning model into our search engine. This model analyzes user queries and generates 
alternative query suggestions, assisting users in refining their search terms and discovering 
additional relevant publications. By leveraging the power of machine learning, we aim to 
continuously improve the search engine's capability to adapt and cater to users' needs.

%\section{Architecture}
% probably add a diagram
\section{Implementation}
\subsection{Ranking Implementation}

The ranking has been implemented using the BM25F algorithm, which is a variant of the BM25 
ranking algorithm that is designed to handle documents with multiple fields. Each field 
contributes to the final score of a document. The following steps outline how the ranking 
process works:

\begin{enumerate}
    \item The \texttt{BM25F} class is defined, taking the inverted index and the total number of 
    documents in the collection as input.
    \item The \texttt{bm25f} method in the \texttt{BM25F} class calculates the BM25F score for a 
    list of documents and a query. It calculates the score for each field (e.g., title, 
    abstract) and then sums up the scores to obtain the final score for each document.
    \item The \texttt{\_idf\_calculation} method calculates the IDF (Inverse Document Frequency) 
    for each query term. IDF is a measure of how important a term is in the collection.
    \item The \texttt{\_tf\_field\_calculation} method calculates the term frequency (TF) for 
    each query term and field. TF is a measure of how frequently a term appears in a document's 
    field.
    \item The \texttt{\_algorith\_parameters} method returns the algorithm parameters (k1 and b) 
    for each field. These parameters control the impact of term frequency and field length in 
    the scoring process.
    \item The \texttt{SearchEngine} class is defined, taking the inverted index and the maximum 
    number of results to return as input.
    \item The \texttt{search} method in the \texttt{SearchEngine} class performs the search 
    operation using the BM25F ranking algorithm. It cleans the query, counts the results, 
    performs boolean search if needed, ranks the documents using BM25F, and calculates the final 
    score for each document based on a combination of the BM25F score and the referenced\_by 
    score.
    \item The \texttt{BooleanInformationRetrieval} class is defined, which performs boolean 
    search by finding the documents that contain all the words in the query. The documents are 
    then ranked by the number of words they contain.
    \item The \texttt{Relevancy} class provides a cosine similarity method to calculate the 
    similarity between two sentences.
    \item The \texttt{sort\_documents} method in the \texttt{SearchEngine} class sorts the 
    documents based on their scores in descending order.
\end{enumerate}

The mathematical formula of the BM25F algorith is:

\begin{multline}
    \text{{score}}(Q, D) = \sum_{i=1}^{n} \text{{IDF}}(q_i) \cdot \\
    \left[ \frac{\sum_{j=1}^{m} W_{f_j} \cdot \text{{TF}}(q_i, F_{f_j}, D) \cdot (k1_{f_j} + 1)}{\sum_{j=1}^{m} W_{f_j} \cdot \text{{TF}}(q_i, F_{f_j}, D) + k1_{f_j} \cdot (1 - b_{f_j} + b_{f_j} \cdot \frac{\text{{length}}(F_{f_j}, D)}{\text{{avg\_length}}(F_{f_j})})} \right]
\end{multline}

Where:
\begin{itemize}
  \item $Q = \{q_1, q_2, ..., q_n\}$ is a query with $n$ terms.
  \item $D$ is a document.
  \item $IDF(q_i)$ is the inverse document frequency of term $q_i$.
  \item $TF(q_i, F_{f_j}, D)$ is the term frequency of term $q_i$ in field $f_j$ of document $D$.
  \item $W_{f_j}$ is the weight of field $f_j$.
  \item $k1_{f_j}$ and $b_{f_j}$ are free parameters, specific to each field $f_j$.
  \item $\text{{length}}(F_{f_j}, D)$ is the length of the field $f_j$ in the document $D$.
  \item $\text{{avg\_length}}(F_{f_j})$ is the average length of field $f_j$ in the collection.
\end{itemize}

This formula calculates the BM25F score for a given document $D$ and query $Q$ by summing 
the contributions of each query term $q_i$, where the contribution is determined by the IDF of 
the term and the weighted term frequency in each field $f_j$. The term frequency is normalized
by the field length and average field length in the collection. The weight of each field $f_j$ is
given by $W_{f_j}$. The parameters $k1_{f_j}$ and $b_{f_j}$ are specific to each field, and they control the 
scaling by term frequency and field length, respectively \cite{rostami2022investigating}. 

In summary, the ranking process involves calculating the IDF, TF, and BM25F \cite{PJAGIF10} 
scores for each 
document and query term, and then combining these scores to obtain the final ranking. The 
process takes into account factors such as term frequency, field length, and document references 
to determine the relevance and ranking of documents.
% TODO add citation and mathematical formula

\subsubsection{Boolean Information Retrieval Algorithm}
The Boolean Information Retrieval algorithm was used to initially narrow down the results. 
If the number of texts containing any of the words found in the query, then we use this 
algorithm to give an initial score to the texts. For each word of the query that contains 
the text, it gets a point, and then a classification is made based on this point. In the end, 
the texts with the highest score are the ones that pass to the next stage of the BM25F algorithm.

\subsubsection{Preprocessor Package}

The preprocessor package provides functionality to clean and preprocess the data before 
performing any analysis or indexing. The package consists of the \texttt{DataCleaner} class, 
which performs various data cleaning operations. Here's a summary of the implemented 
functionality:

\begin{itemize}
    \item \texttt{clean\_data(data)}: Cleans the given data by performing several data cleaning 
    operations. It removes HTML tags, converts the data to lowercase, tokenizes the data into a 
    list of words, and removes stopwords and punctuation from the data. The cleaned data is 
    returned as a list of words.
    \item \texttt{\_\_html\_strip(data)}: Removes HTML tags from the given data.
    \item \texttt{\_\_lower\_string(data)}: Converts the given data to lowercase.
    \item \texttt{\_\_tokenizer(data)}: Tokenizes the given data into a list of words.
    \item \texttt{\_\_punctuation\_deletion(data)}: Removes stopwords and punctuation from the 
    given data.
\end{itemize}

The \texttt{DataCleaner} class utilizes the NLTK library for various operations. It downloads 
the necessary NLTK resources, such as stopwords and tokenizers, during initialization.

The data cleaning operations performed by the \texttt{DataCleaner} class help in preparing the 
data for further analysis or indexing. It removes unnecessary HTML tags, converts the data to 
lowercase, tokenizes the data into words, and removes common stopwords and punctuation. The 
cleaned data can then be used for various natural language processing tasks.

Overall, the preprocessor package provides essential functionality for cleaning and 
preprocessing data, ensuring that it is in a suitable format for further analysis or indexing.

\subsection{Indexer Package}

The indexer package provides functionality to create and manage indexes for a MongoDB database. 
It consists of the \texttt{Reader} and \texttt{IndexCreator} classes, along with supporting 
classes \texttt{Metadata} and \texttt{InfoClass}.

\subsubsection{Reader}

The \texttt{Reader} class handles reading operations from a MongoDB database. It provides 
methods to retrieve a collection, read all documents from a collection, and read a limited 
number of documents from a collection. Here are the key methods of the \texttt{Reader} class:

\begin{itemize}
    \item \texttt{\_\_init\_\_(connection\_string)}: Initializes a \texttt{Reader} object with 
    the given MongoDB connection string.
    \item \texttt{get\_collection(db\_name, collection\_name)}: Retrieves a collection from the 
    MongoDB database.
    \item \texttt{read\_collection(collection)}: Reads all documents from the provided 
    collection.
    \item \texttt{read\_limited\_collection(collection, limit)}: Reads a limited number of 
    documents from the provided collection.
\end{itemize}

\subsubsection{Metadata}

The \texttt{Metadata} class handles metadata operations related to the index. It maintains 
information about document's field lengths, average lengths, and references. Here are the key methods of 
the \texttt{Metadata} class:

\begin{itemize}
    \item \texttt{\_\_init\_\_(db\_name, metadata\_collection)}: 
    Initializes the \texttt{Metadata} object with the given MongoDB database name and collection name.
    \item \texttt{update\_doc\_num()}: Updates the total number of documents.
    \item \texttt{add\_doc\_length\_field(doc\_id, length, field)}: Adds the length of a 
    specific field in a document.
    \item \texttt{increase\_average\_length(length, field)}: Increases the average length of a 
    specific field.
    \item \texttt{calculate\_average\_length()}: Calculates the average length for each field.
    \item \texttt{add\_referenced\_by(doc\_id, referenced)}: Adds the number of times a document 
    is referenced by other documents.
    \item \texttt{normalize\_referenced\_by()}: Normalizes the referenced\_by values.
    \item \texttt{set\_total\_docs(num)}: Sets the total number of documents.
    \item \texttt{load()}: Loads the metadata from the MongoDB collection.
    \item \texttt{save()}: Saves the metadata to the MongoDB collection.
\end{itemize}

\subsubsection{InfoClass}

The \texttt{InfoClass} class represents information about a specific document in the index. It 
stores the type, position, and document ID associated with the information. The class provides a 
method to convert the instance into a dictionary representation.

\subsubsection{IndexCreator}

The \texttt{IndexCreator} class handles the creation and loading of the index. It utilizes the 
\texttt{TrieIndex} class from the \texttt{distributed} module for indexing. Here are the key 
methods of the \texttt{IndexCreator} class:

\begin{itemize}
    \item \texttt{\_\_init\_\_(db, metadata\_collection, db\_name='M151', 
    index\_collection='Index')}: Initializes an instance of \texttt{IndexCreator} with the given 
    parameters.
    \item \texttt{create\_load\_index()}: Creates or loads the index. It creates the index if 
    the collection is empty or loads the existing index.
    \item \texttt{node\_adder(\_id, cleaned\_words, \_type)}: Adds nodes to the index.
\end{itemize}

The \texttt{IndexCreator} class utilizes the \texttt{DataCleaner} class from the preprocessor 
package for cleaning the data before indexing. It also interacts with the MongoDB database using 
the \texttt{db.connection} module.

Overall, the indexer package provides functionality to create, load, and manage indexes for a 
MongoDB database. It incorporates data cleaning, metadata management, and indexing operations, 
ensuring efficient retrieval and analysis of the indexed data.


\subsection{APIRequester Package}

The APIRequester package provides functionality for sending HTTP requests to an API. It includes 
the \texttt{APIRequester} class, which handles authentication and sending POST requests to the 
API endpoints.


The \texttt{APIRequester} class is responsible for creating and sending HTTP requests to the 
API. It requires the base URL, username, and password for authentication. Here are the key 
methods of the \texttt{APIRequester} class:

\begin{itemize}
    \item \texttt{\_\_init\_\_(base\_url, username, password)}: Initializes an 
    \texttt{APIRequester} object with the base URL, username, and password.
    \item \texttt{post\_update\_config\_endpoint(data)}: Sends a POST request to the 
    'update\_config' endpoint of the API with the provided configuration data. Returns the 
    parsed JSON response if the request is successful, otherwise raises an exception with the 
    corresponding status code.
\end{itemize}

The \texttt{APIRequester} class utilizes the \texttt{http.client} module to establish a 
connection with the API and send HTTP requests. It also utilizes the \texttt{base64} and 
\texttt{json} modules for authentication and parsing JSON responses, respectively. 
The \texttt{urllib.parse} module is used to parse the base URL.

Overall, the APIRequester package provides a convenient way to authenticate and send POST 
requests to an API. It abstracts away the low-level details of establishing connections and 
handling authentication, allowing users to focus on interacting with the API and processing the 
responses.

\subsection{API Server}

The API server is implemented using the FastAPI framework. It provides several endpoints for 
handling different types of requests. The server supports basic authentication and uses a 
configuration manager and request wrapper to handle configuration updates and API requests, 
respectively.

\subsubsection{API Server Setup}

The server is set up using the FastAPI framework. It initializes the FastAPI application, sets 
up logging, and configures CORS middleware to handle cross-origin requests. The configuration 
file paths are defined, and instances of the ConfigManager, RequestWrapper, and IniConfig 
classes are created.

\subsubsection{Authentication}

The server supports basic authentication using the provided credentials. 
The \texttt{get\_current\_username} function validates the provided username and password 
against the configured credentials. If the credentials are incorrect, an HTTPException with 
status code 401 (Unauthorized) is raised.

\subsubsection{Endpoints}

The API server provides the following endpoints:

\begin{itemize}
    \item \texttt{GET /search\_ids/\{query}\}: Retrieves a list of document IDs matching the 
    provided query.
    \item \texttt{GET /alternate\_queries/\{query\}}: Retrieves a list of alternate queries 
    based on the provided query.
    \item \texttt{POST /fetch\_data/}: Fetches data for the specified document IDs.
    \item \texttt{POST /update\_config}: Updates the configuration and saves it.
\end{itemize}

The \texttt{search\_ids} endpoint calls the \texttt{search\_ids} method of the 
\texttt{RequestWrapper} class asynchronously, using a ThreadPoolExecutor to handle the request 
concurrently. The result is returned as a list of document IDs.

The \texttt{alternate\_queries} endpoint returns a static list of alternate queries.

The \texttt{fetch\_data} endpoint calls the \texttt{fetch\_data} method of the 
\texttt{RequestWrapper} class asynchronously, using a ThreadPoolExecutor to handle the request 
concurrently. The result is returned as a list of dictionaries containing the fetched data for 
each document.

The \texttt{update\_config} endpoint reads the updated configuration data from the HTTP request, 
saves it using the \texttt{ConfigManager}, and updates the \texttt{neighbour\_nodes} property of 
the \texttt{RequestWrapper}.

\subsubsection{Running the Server}

The server is run using the \texttt{uvicorn} package, which serves the FastAPI application. The 
host and port are read from the configuration file.

Overall, the API server provides a RESTful API with authentication and support for concurrent 
requests. It allows users to search for document IDs, fetch data, and update the configuration 
of the system.

\subsection{The \texttt{distributed} Package}

The \texttt{distributed} package offers functionality for handling JSON documents stored in a 
distributed manner across multiple files. It includes the following key features:

\subsubsection{Shard Calculation}
The package provides a function \texttt{get\_shard(doc\_id, num\_shards)} to determine the shard 
ID based on the document ID and the total number of shards.

\subsubsection{FastJsonLoader Class}
This class manages the loading and management of JSON documents. It includes the following 
methods:

\begin{itemize}
    \item \texttt{\_\_init\_\_(...)}: The class constructor allows for setting parameters such 
    as the folder path containing the JSON files, the number of documents per file, the node ID, 
    the total node count, and the MongoDB collection for storing document IDs.
    
    \item \texttt{load\_ids()}: Loads the document IDs from MongoDB.
    
    \item \texttt{save\_ids()}: Saves the document IDs to MongoDB.
    
    \item \texttt{load\_documents()}: Reads JSON files from the specified folder and organizes 
    the documents into compressed data files based on the desired number of documents per file. 
    It also handles metadata about the document files.
    
    \item \texttt{get\_data(doc\_ids, sort\_by\_doc\_id=False)}: Retrieves data for the given 
    document IDs. It can also sort the results based on the order of the provided doc\_ids.
    
    \item \texttt{insert\_documents(new\_documents)}: Inserts new documents into the JSON files.
    
    \item \texttt{delete\_documents(doc\_ids\_to\_delete)}: Deletes documents from the JSON 
    files.
    
    \item \texttt{count\_documents\_in\_folder()}: Counts the total number of documents present 
    in the specified folder.
    
    \item \texttt{get\_all\_documents()}: Retrieves all documents from the JSON files.
    
    \item \texttt{id\_file\_exists()}: Checks if the document IDs exist in MongoDB.
\end{itemize}

This package provides a convenient and efficient way to manage and work with distributed JSON 
documents across multiple files.

\subsection{Implementation of a Distributed Bibliographic Search Engine}

In this article, we will explore the deployment of a bibliographic search engine that has been 
implemented in a distributed manner. The search engine is designed to handle large volumes of 
bibliographic data and provide efficient search capabilities. The system consists of multiple 
nodes that work together to distribute the workload and improve performance. Here, we will 
discuss the implementation details of the \texttt{DistributedNode} class, which represents an 
individual node in the distributed system.

\subsection*{\texttt{DistributedNode} Class Overview}

The \texttt{DistributedNode} class is responsible for managing the operations of a single node 
in the distributed bibliographic search engine. It handles incoming requests, performs document 
loading and indexing, and facilitates distributed search operations. Let's go through the key 
components and methods of this class to understand its functionality.


\subsubsection*{Initialization}

The \texttt{\_\_init\_\_} method initializes a \texttt{DistributedNode} object with various 
parameters. These parameters include the node's ID, host address, port, file paths, database 
information, SSL encryption details, API credentials, and document indexing metadata.


\subsubsection*{Handling Requests}

The \texttt{handle\_request} method processes incoming requests and generates appropriate 
responses. It receives a JSON-encoded request, extracts the action type, and performs the 
corresponding operation. The supported actions include heartbeat, document loading, index 
creation/loading, data retrieval, updating node status, ID-based search, setting the starting 
document ID, inserting new documents, retrieving the configuration, and setting the leader.

\subsubsection*{Running the Node}

The \texttt{run} method starts the server and listens for incoming connections on the specified 
host and port. It creates a new thread for each accepted connection to handle the request 
asynchronously. The server continuously runs in a loop, accepting and processing requests.


\subsubsection*{Heartbeat and Initialization}

The \texttt{check\_heartbeats} method is used for checking the heartbeats of the neighboring 
nodes and initiating the document loading process. It sends heartbeat requests to all nodes and 
verifies their status. If the current node is the leader and it is the first boot, it starts a 
separate thread to continuously check the heartbeats. On subsequent boots, the method retrieves 
the configuration from other nodes and starts the document loading and index creation process.

\subsubsection*{Interacting with Neighboring Nodes}

The \texttt{update\_alive\_nodes} method updates the list of alive nodes by sending heartbeat 
requests to each node. It iterates through the neighboring nodes and sends heartbeat requests to 
determine their status.

The \texttt{notify\_nodes\_of\_leader} method notifies the neighboring nodes about the new 
leader in the system. It sends a request to each node, informing them of the current leader's 
details.

\subsubsection*{Data Retrieval and Search}

The \texttt{handle\_get\_data} method handles the retrieval of data based on IDs. It splits the 
IDs based on the node count and distributes the requests to the respective nodes. It uses 
multithreading to perform data retrieval in parallel and aggregates the results.

The \texttt{search\_ids} method performs a search operation based on a query string. Similar to 
data retrieval, it distributes the search requests across nodes using multithreading and 
combines the results.

\subsubsection*{Forwarding Requests}

The \texttt{forward\_request} method forwards a request to a specified node. It takes the node 
details and request data as input and sends the request to the designated node. It can also 
forward requests to the leader based on the \texttt{to\_leader} parameter.

\subsubsection*{Configuration Management}

The \texttt{get\_leader} method retrieves the leader node from the neighbor nodes based on their 
IDs. It returns the node with the lowest ID as the leader.


The \texttt{update\_config} method updates the local configuration file with the current 
neighbor node information. It writes the updated configuration to the file.


The deployment of a distributed bibliographic search engine requires the coordination and 
collaboration of multiple nodes. The \texttt{DistributedNode} class plays a crucial role in 
managing the operations of an individual node within the system. It handles requests, 
facilitates distributed document loading and indexing, and supports efficient data retrieval and 
search capabilities. By leveraging the power of a distributed architecture, the search engine 
can efficiently process large volumes of bibliographic data and provide accurate search results.

\section{Deployment}
For the deployment of our bibliographic search engine, the following approach has been followed.
The front-end of our system has been developed using the Spring Boot framework.
The back-end functionalities have been implemented in Python, aiming for efficient data storage, 
retrieval, and processing. 

To ensure the quality and reliability of our codebase, we have implemented automated testing 
procedures. Using the GitHub Actions of the repository, 
continuous integration and deployment pipelines have been set up that run a series of unit tests 
on each commit. This ensures that any issues or bugs are identified early on, allowing for 
prompt resolution and maintaining a stable system.

The distributed database approach was adopted, utilizing a total of eight nodes. 
This distributed setup allowed us to efficiently handle the immense size and complexity of the 
Crossref dataset. In the appendix~\ref{apx:hardware-specs}, please find detailed hardware 
specifications for each node, including information such as CPU, memory, and storage capacity. 
By strategically distributing the workload across multiple nodes, we aimed to achieve enhanced 
performance, improved fault tolerance, and the ability to handle a high volume of user queries 
simultaneously. 

\section{Observations}

\subsection{Indexer}
In order to be able to search the documents we store and serve the queries that come, we chose 
to use the structure of the inverted index. An inverted index is a data structure used to create 
a full-text search index, enabling fast, keyword-based searches. It is one of the most critical 
components of search engines.

The term $inverted$ comes from the way the index is built: instead of listing, for each document 
the terms it contains, it lists for each term, the documents that contain it, thus "inverting" 
the relationship between documents and terms. Inverted indexes are used because they allow very 
fast keyword-based searches. Once the inverted index is built, finding all documents containing 
a particular keyword is as simple as looking up the keyword in the index, which is a very fast 
operation. Without an inverted index, a keyword-based search would require scanning through all 
documents in the collection, which is impractical for large document collections.

To reduce the search time for a single word, the inverted index must be in main memory, so that 
we do not have to access the disk every time. In order to reduce the size of the index and fit it 
entirely in memory, we built it using a $Trie$ data structure. So we observed a dramatic reduction 
in the size of the inverted index. Also a side benefit observed using this structure was an 
increase in search speed.

\subsection{BM25F Parallelization}
To increase the performance of the BM25F algorithm we tried to parallelize the calculation of the 
scores of the documents. We studied two basic techniques, to parallelize computation at process 
level and at thread level. In both cases the number of documents of each node of the system, was 
divided into processes or threads, the scores of each chunk were calculated in parallel, and at 
the end the results were collected in a common structure that contained all the texts and their 
scores. But through experiments, both with the number of documents, but also with the number of 
processes or threads, we did not see an improvement in the performance. Below we explain the 
possible reasons why we did not see a performance improvement for each technique:

\begin{itemize}
    \item {\texttt{Process} Level Parallelization}: Multiprocessing in Python has an overhead 
    associated with it due to the need to spin up new processes, communicate between them 
    (which involves pickling and unpickling data), and then collect the results. This overhead 
    can be substantial, especially for smaller tasks or faster computations where the time 
    spent setting up and managing the processes can outweigh the benefits of parallel execution.
    BM25F algorithm is not be sufficiently complex or time-consuming to benefit from parallel 
    execution, particularly given the overheads associated with multiprocessing. The overheads 
    may include:

    \begin{itemize}
        \item The time taken to start and stop processes.
        \item The time taken to divide the task into chunks.
        \item The time taken to communicate data between processes (serialization and deserialization).
    \end{itemize}

    \item {\texttt{Thread} Level Parallelization}: Due to Python's GIL BM25F algorithm couldn't 
    parallelized in thread level. Python's Global Interpreter Lock, or GIL, is a mutex (or a lock) 
    that allows only one thread to execute at a time in a single process. This is because CPython, 
    the reference implementation of Python, is not thread-safe. The GIL is used to prevent 
    multiple native threads from executing Python bytecodes at once, which can cause 
    inconsistencies and problems with Python objects.

    The GIL makes single-threaded programs fast and efficient, but it can significantly slow 
    down multi-threaded CPU-bound programs. It is because, despite having multiple threads 
    within a process, due to the GIL, only one thread can execute at a time, negating the 
    benefits of multi-core processors. Essentially, the GIL leads to essentially single-threaded 
    execution, even on multi-threaded Python programs.
\end{itemize}

% \newpage
% \printbibliography{bib}

\newpage
\appendix
\section{Hardware Specifications}
\label{apx:hardware-specs}
\begin{table}[htbp]
    \centering
    \caption{RAM Information}
    \begin{tabular}{ll}
      \toprule
      \textbf{Property} & \textbf{Value} \\
      \midrule
      \multicolumn{2}{l}{\textbf{Firmware}} \\
      Vendor & DigitalOcean \\
      Version & 20171212 \\
      Date & 12/12/2017 \\
      Size & 96KiB \\
      Capabilities & virtualmachine \\
      \midrule
      \multicolumn{2}{l}{\textbf{Memory}} \\
      Physical ID & 1000 \\
      Size & 8GiB \\
      Capabilities & ecc \\
      Configuration & errordetection=multi-bit-ecc \\
      \midrule
      \multicolumn{2}{l}{\textbf{Memory Device}} \\
      Array Handle & 0x1000 \\
      Size & 8 GB \\
      Form Factor & DIMM \\
      Set & None \\
      Locator & DIMM 0 \\
      Type & RAM \\
      Manufacturer & QEMU \\
      \bottomrule
    \end{tabular}
  \end{table}

  \newpage
  \pagestyle{fancy} % Apply the modified page style to the current page
  \begin{table}[htbp]
    \centering
    \caption{System Information}
    \begin{tabular}{ll}
      \toprule
      \textbf{Property} & \textbf{Value} \\
      \midrule
      \multicolumn{2}{l}{\textbf{Computer}} \\
      Product & Droplet \\
      Vendor & DigitalOcean \\
      Version & 20171212 \\
      Width & 64 bits \\
      Capabilities & smbios-2.8 dmi-2.8 smp vsyscall32 \\
      Configuration & boot=normal family=DigitalOcean\_Droplet \\
      \midrule
      \multicolumn{2}{l}{\textbf{Motherboard}} \\
      Product & Droplet \\
      Vendor & DigitalOcean \\
      Version & 20171212 \\
      \midrule
      \multicolumn{2}{l}{\textbf{Firmware}} \\
      Description & BIOS \\
      Vendor & DigitalOcean \\
      Physical ID & 0 \\
      Version & 20171212 \\
      Date & 12/12/2017 \\
      Size & 96KiB \\
      Capabilities & virtualmachine \\
      \midrule
      \multicolumn{2}{l}{\textbf{CPU}} \\
      Description & CPU \\
      Product & DO-Regular \\
      Vendor & Intel Corp. \\
      Physical ID & 400 \\
      Bus Info & cpu@0 \\
      Version & 6.63.2 \\
      Slot & CPU 0 \\
      Size & 2GHz \\
      Capacity & 2GHz \\
      Width & 64 bits \\
      Configuration & cores=4 enabledcores=4 microcode=1 threads=1 \\
      \midrule
      \multicolumn{2}{l}{\textbf{Memory}} \\
      Description & System Memory \\
      Physical ID & 1000 \\
      Size & 8GiB \\
      Capabilities & ecc \\
      Configuration & errordetection=multi-bit-ecc \\
      \midrule
      \multicolumn{2}{l}{\textbf{Memory Device}} \\
      Description & DIMM RAM \\
      Vendor & QEMU \\
      Physical ID & 0 \\
      Slot & DIMM 0 \\
      Size & 8GiB \\
      \midrule
      \multicolumn{2}{l}{\textbf{Host Bridge}} \\
      Product & 440FX - 82441FX PMC [Natoma] \\
      Vendor & Intel Corporation \\
      Physical ID & 100 \\
      Width & 32 bits \\
      Clock & 33MHz \\
      \midrule
      \bottomrule
    \end{tabular}
  \end{table}
  \newpage
  
  \begin{table}[H]
    \centering
    \caption{CPU Information}
    \begin{tabular}{l l}
      \toprule
      \textbf{Property} & \textbf{Value} \\
      \midrule
      \textbf{Architecture} & x86\_64 \\
      CPU op-mode(s) & 32-bit, 64-bit \\
      Address sizes & 40 bits physical, 48 bits virtual \\
      Byte Order & Little Endian \\
      \midrule
      \textbf{CPU(s)} & 4 \\
      On-line CPU(s) list & 0-3 \\
      \textbf{Vendor ID} & GenuineIntel \\
      Model name & DO-Regular \\
      CPU family & 6 \\
      Model & 63 \\
      Thread(s) per core & 1 \\
      Core(s) per socket & 4 \\
      Socket(s) & 1 \\
      Stepping & 2 \\
      BogoMIPS & 3990.62 \\
      \midrule
      \textbf{Virtualization features} & \\
      Virtualization & VT-x \\
      Hypervisor vendor & KVM \\
      Virtualization type & full \\
      \midrule
      \textbf{Caches (sum of all)} & \\
      L1d & 128 KiB (4 instances) \\
      L1i & 128 KiB (4 instances) \\
      L2 & 16 MiB (4 instances) \\
      \midrule
      \textbf{NUMA} & \\
      NUMA node(s) & 1 \\
      NUMA node0 CPU(s) & 0-3 \\
      
      \bottomrule
    \end{tabular}
  \end{table}

\end{document}



